% !TeX spellcheck = en_US

\documentclass[a4paper,11pt]{book}

\usepackage{parskip}
\usepackage{graphicx}
\usepackage{amsmath}
\usepackage{amssymb}
\usepackage{amsfonts}
%\usepackage[utf8]{inputenc} %fuers ue
\usepackage[latin1]{inputenc}
\usepackage[T1]{fontenc} 
\usepackage{siunitx}
%\usepackage[ngerman]{babel} %rechtschreibung?, silbentrennung
\usepackage{float}
\usepackage{caption}
\usepackage[font=small, format=plain, labelfont=bf, textfont=it]{caption} %format figure captions
\usepackage{textcomp} %extra symbole
\usepackage{fancyhdr} %Kopfzeile fancy package bestimmt seitenlayout
\usepackage{subfigure}
%\usepackage{setspace} %fuer Absatzabstaende
\usepackage[numbers,sort&compress]{natbib}
\usepackage[raggedright]{titlesec}
\usepackage{lmodern}
\usepackage{array,multirow,graphicx}
\usepackage{url}
\usepackage{hyperref} %erzeugt im pdf links
\usepackage{scrextend}

\setlength{\belowcaptionskip}{-10pt}

%\singlespacing

\setlength{\headheight}{15.2pt} %Seitenraender und Format
\textwidth=15cm
\textheight=21cm %vorher 21cm (war mal 22cm)
\topmargin=10pt
\oddsidemargin=1cm
\evensidemargin=0cm

\setcounter{secnumdepth}{4} %Tiefe der Nummerierung
\setcounter{tocdepth}{4}
\setlength{\parskip}{0.5cm} %Abstand beim Absatz (einruecken)





%\singlespacing

\fontencoding{T1}

%standard graphics path
\graphicspath{{./figures/}}

\begin{document}

\hyphenation{photo-lu-mi-nes-cence}
\hyphenation{electro-lu-mi-nes-cence}
\hyphenation{nano-wire}
\hyphenation{nano-wires}
\hyphenation{hetero-struc-ture}
\hyphenation{hetero-struc-tures}
\hyphenation{crystal-lite}
\hyphenation{down-scaling}
\hyphenation{Kos-ten}
\hyphenation{Leistungs-stei-ger-ung}
\hyphenation{phy-si-ka-lisch}
\hyphenation{Bauteil-konzepte}
\hyphenation{Bau-teile}
\hyphenation{Struk-tu-ren}
\hyphenation{Nano-dr\"ah-ten}
\hyphenation{Nano-dr\"ahte}
\hyphenation{opto-elektronische}
\hyphenation{Ladungs-tr\"ager-mobilit\"at}
\hyphenation{zahl-reiche}
\hyphenation{Hetero-strukturen}
\hyphenation{ge-wach-sen}
\hyphenation{Diffusions-koeffizienten}
\hyphenation{Seg-men-te}
\hyphenation{Rat-schl\"age}
\hyphenation{litho-graph-isches}
\hyphenation{Feldeffekt-transistoren}
\hyphenation{Fertigungs-prozesses}
\hyphenation{Emmerich}
\hyphenation{Bertagnolli}
\hyphenation{Nijholt}

\bibliographystyle{unsrtnat}%Style des Literaturverzeichnisses

\pagestyle{empty}
% !TeX spellcheck = de_DE

\begin{titlepage}

\begin{center}

\begin{figure}[ht]
	\centering
		\includegraphics[width=0.50\textwidth]{figures/TULogo.png}
	\label{fig:TULogo}
\end{figure}



\vfill

\LARGE{DIPLOMARBEIT}\\

\vfill

\LARGE{\textbf{XXXXXXXXXXXXXXXXXXXXXXXXX}}\\
\vfill

\large
ausgef\"uhrt zum Zwecke der Erlangung des\\
akademischen Grades eines 
Diplom - Ingenieurs\vfill

an der Technischen Universit\"at Wien \\
Institut f\"ur Festk\"orperelektronik\\

\vfill

unter der Leitung von\\
~\\
\textbf{Ao.Univ.Prof. Dipl.-Ing. Dr.techn. Alois Lugstein} \\
\vfill

durch\\
\textbf{XXX XXX, BSc} \\
Matr.-Nr. XXXX\\
XXXXXXX\\
XXXXX Wien\\

\vfill

\vfill

\vfill

\vfill

\vfill 

\small{Wien, XXXXXX 201X}


\end{center}
\end{titlepage}

\clearpage{\pagestyle{empty}\cleardoublepage}

\pagenumbering{roman}\setcounter{page}{1}%beginne mit roemischen seitenzahlen
\pagestyle{plain}
%\onehalfspacing% 1.5 Zeilenabstand
%\doublespacing %2 Zeilenabstand


\phantomsection

\vfill

\section*{\centering{Statutory Declaration}}

I declare, that I have authored the present work independently according to the code of conduct, that I have not used other than the declared sources and that I have explicitly marked all material quoted either literally or by content from the used sources. This work was not yet submitted to any examination procedure neither in Austria, nor in any other country.
\vspace{3cm}

\section*{\centering{Erkl\"arung zur Verfassung der Arbeit}}

Hiermit erkl\"are ich, dass die vorliegende Arbeit gem\"a{\ss} dem Code of Conduct - Regeln zur Sicherung guter wissenschaftlicher Praxis - ohne unzul\"assige Hilfe Dritter und ohne Benutzung anderer als der angegebenen Hilfsmittel, angefertigt wurde. Die aus anderen Quellen direkt oder indirekt \"ubernommenen Daten und Konzepte sind unter Angabe der Quelle gekennzeichnet. Die Arbeit wurde bisher weder im In- noch im Ausland in gleicher oder in \"ahnlicher Form in anderen Pr\"ufungsverfahren vorgelegt.
\vspace{4cm}

\indent \hspace{0.3cm} Wien, XX. XX 201X \hspace{2.9cm} ...................................................\\
\indent \hspace{9.5cm} XXXXXXXXXXXXX

\deffootnote[2em]{2em}{1em}{\textsuperscript{\thefootnotemark}\,}

\phantomsection
% !TeX spellcheck = en_En

%vertikaler Leerraum.            Befehle von Johannes 
%\vspace*{1.5cm}%vorher 2.2cm
%\noindent%kein Einzug
%{\Huge {\bf Abstract}}


%\vspace*{0.6cm} %vorher 1.6cm
\section*{Abstract}


%COMPONENT #1: Study background and significance
%COMPONENT #2: Components of your research strategy
%COMPONENT #3: Findings
%COMPONENT #4: Conclusions


\newpage

% !TeX spellcheck = en_GB

\section*{Kurzfassung}


\phantomsection
% !TeX spellcheck = en_US

\vspace*{1.3 cm} 

\section*{\LARGE{Acknowledgement}}
\label{sec:Acknowledgement}


%\singlespacing

%\tableofcontents%  Inhaltsverzeichnis
%\doublespacing

\renewcommand{\baselinestretch}{1.08}\normalsize
\tableofcontents
\renewcommand{\baselinestretch}{1.0}\normalsize
\clearpage{\pagestyle{empty}\cleardoublepage}

%\singlespacing

\newpage
%\thispagestyle{empty} erzeugt Leerseite ohne Seitenzahl etc.
%\section*{}

\fancypagestyle{plain}{
 \fancyhf{}				%leert die h=head und f=foot bereiche
 \renewcommand{\headrulewidth}{0pt}
 \fancyfoot[C]{\thepage}
 %\fancyhead[R]{\thepage} Seitenzahlen rechts oben
}						%redefine plain in fancyhdr package weil: chapter beginnseiten werden immer in plain angezeigt
\pagestyle{fancy} 		%ist der Seitenstil der normalen Seiten
\lhead[\leftmark]{}
\rhead[]{}%\thepage} 		Erstellt seitenzahlen rechts oben
%\cfoot{} 				Loescht die Seitenzahlen unten in der Mitte
%chapters

\pagenumbering{arabic}
\setcounter{page}{1}

%\onehalfspacing
%\doublespacing

% !TeX spellcheck = en_US

\chapter{Introduction}

\clearpage{\pagestyle{empty}\cleardoublepage}





  

% !TeX spellcheck = en_US

\chapter{Theory}
\label{chapter:Theory}

\clearpage{\pagestyle{empty}\cleardoublepage}

% !TeX spellcheck = en_US

\chapter{Experimental Techniques}
\label{chapter:ExperimentalTechniques}

\clearpage{\pagestyle{empty}\cleardoublepage}

% !TeX spellcheck = en_US

\chapter{Results and Discussion}
\label{chapter:ResultsAndDiscussion}


\clearpage{\pagestyle{empty}\cleardoublepage}

% !TeX spellcheck = en_US

\chapter{Summary and Outlook}
\label{chapter:Conclusion}



\clearpage{\pagestyle{empty}\cleardoublepage}


\phantomsection
\addcontentsline{toc}{chapter}{List of Figures}
\listoffigures


\clearpage
\phantomsection
\addcontentsline{toc}{chapter}{List of Abbreviations}
\lhead[LIST OF ABBREVIATIONS]{}
% !TeX spellcheck = en_US

\chapter*{List of Abbreviations}
\label{sec:ListOfAbbreviations}

\begin{tabular}{l l}

Al & Aluminum \\
Al$_2$O$_3$ & Aluminum Oxide \\
ALD & Atomic Layer Deposition \\
Ar & Argon \\
Au & Gold \\
BHF & Buffered Hydrofluoric Acid\\
C$_3$H$_9$Al & Trimethylaluminium \\
CMOS & Complementary Metal-Oxide-Semiconductor \\
CVD & Chemical Vapor Deposition \\
Co & Cobalt \\
Cr & Chrome \\
Cu & Copper \\
DI & Deionized Water \\
%DIBL & Drain-Induced Barrier Lowering \\
DRAM & Dynamic Random Access Memory \\
EBL & Electron Beam Lithography \\
ESD & Electrostatic Discharge \\
Fcc & Face Centered Cubic \\
FET & Field-Effect Transistor \\
FFT & Fast Fourier Transformation \\
FIB & Focused Ion Beam \\
GAA & Gate-All-Around \\
GaAs & Gallium Arsenide \\
GaN & Gallium Nitride \\
%GIDL & Gate-Induced Drain Leakage \\
Ge & Germanium \\
GeO & Germanium Monoxide \\
GeO$_2$ & Germanium Dioxide \\
GH$_4$ & Germane \\
HAADF & High-Angle Annular Dark Field \\

\end{tabular}

\newpage

\begin{tabular}{l l}

H$_2$O & Water \\
HF & Hydrofluoric Acid \\
HfO$_2$ & Hafnium Oxide \\
HI & Hydroiodic Acid \\
HRTEM & High-resolution Transmission Electron Microscopy \\
II & Impact Ionization \\
I/V & Current-Voltage \\
InAs & Indium Arsenide \\
LN$_2$ & Liquid Nitrogen \\
MOSFET & Metal-Oxide-Semiconductor Field-Effect Transistor \\
Mn & Manganese \\
N & Nitrogen \\
Ni & Nickel \\
NDR & Negative Differential Resistance \\ 
NW & Nanowire \\
PCB & Printed Circuit Board \\
PMMA & Polymethylmethacrylat \\
Pt & Platinum \\
QPC & Quantum Point Contact \\
RTA & Rapid Thermal Annealing \\
SSD & Solid-State-Drive \\
SEM & Scanning Electron Microscopy \\
%SET & Single Electron Transistor \\
Si & Silicon\\
SMUs & Source Measure Units \\
Sn & Tin \\
STEM & Scanning Transmission Electron Microscopy \\
Ti & Titanium \\
TiO$_2$ & Titanium Oxide \\
VLS & Vapor-Liquid-Solid \\
VSS & Vapor-Solid-Solid \\
VSU & Voltage Source Unit \\
ZrO$_2$ & Zirconium oxide \\

\end{tabular}


\clearpage
\phantomsection
\addcontentsline{toc}{chapter}{List of Symbols}
\lhead[LIST OF SYMBOLS]{}
% !TeX spellcheck = en_US

\chapter*{List of Symbols}
\label{sec:ListOfSymbols}

\begin{tabular}{l l}

a$_B^*$ & Exciton Bohr Radius \\ 
D & Diffusion Coefficient \\
$\tilde{D}$ & Interdiffusion Coefficient \\
$\nabla{C}$ & Concentration Gradient Vector \\
%E & Energy \\
e & Electronic Charge \\
$E_A$ & Activation Energy \\
E$_C$ & Charging Energy \\
E$_F$ & Fermi Energy \\
G & Conductance \\
g$_S$ & Spin Degeneracy \\
g$_V$ &  Valley Degeneracy \\
%$\Gamma_E$ & Level Spacing \\
$h$, $\hbar$ & Planck Constant,  Reduced Planck Constant \\
%h$\Gamma$ & Level Broadening \\
%$\hbar$ & Reduced Planck Constant \\
I & Current \\
J & Flux of Diffusing Quantities \\
k & Wavenumber \\
$k_B$ & Boltzmann Constant \\
L & Gate Length \\
$l$ & Length \\
$l_m$ & Scattering Mean Free Path \\
$\lambda_F$ & Fermi Wavelength \\
M & Number of Modes \\
$m^*$ & Effective Mass \\
$\mu$ & Electrochemical Potential \\
n & Electron Density \\
%....
$N_A$, $N_B$ & Fractional Concentrations \\
$\sigma$ & Conductivity \\
%R$_{TS}$, R$_{TD}$ & Tunneling Resistance \\
T & Temperature \\
$\tau_m$ & Momentum Relaxation Time\\
%T$_K$ & Kondo Temperature \\
W & Gate Width \\

\end{tabular}

%\begin{tabular}{l l}
%	
%$N_A$, $N_B$ & Fractional Concentrations \\
%$m^*$ & Effective Mass \\
%$l_m$ & Scattering Mean Free Path \\
%$\sigma$ & Conductivity \\
%%R$_{TS}$, R$_{TD}$ & Tunneling Resistance \\
%$\tau_m$ & Momentum Relaxation Time\\
%T & Temperature \\
%%T$_K$ & Kondo Temperature \\
%W & Gate Width \\
%
%\end{tabular}





\clearpage
%\phantomsection
%\addcontentsline{toc}{chapter}{Process Parameters}
\lhead[PROCESS PARAMETERS]{}
% !TeX spellcheck = en_US

\appendix\chapter{Process Parameters}
\label{chapter:ProcessParameters}



\clearpage{\pagestyle{empty}\cleardoublepage}

\clearpage 
\phantomsection
\addcontentsline{toc}{chapter}{Bibliography}
\lhead[\leftmark]{}
\bibliography{library}
\clearpage{\pagestyle{empty}\cleardoublepage}
\end{document}


\makeindex


